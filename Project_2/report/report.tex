\documentclass{article}

\usepackage{parskip}
\usepackage[a4paper, top=2.5cm, bottom=2.5cm, left=2cm, right=2cm]{geometry}
\usepackage{amsmath}     % Math formulas
\usepackage{booktabs}    % High-quality tables
\usepackage{graphicx}    % Images (for placeholders)
\usepackage{placeins}

\title{\Huge Numerical Solution of 2D Heat Conduction}
\author{\normalsize Yifan Zhang  2025251018}
\date{\normalsize \today}

\begin{document}
\maketitle

\section*{Problem Statement}

\section*{Solution}

\subsection*{Discretization}

The standard five-point difference scheme is used to
discretize the equation.

We define that:
$$-\frac{1}{h^2} \left[ k_{i+\frac{1}{2},j}(u_{i+1,j}-u_{i,j}) - k_{i-\frac{1}{2},j}(u_{i,j}-u_{i-1,j}) \right] \\
-\frac{1}{h^2} \left[ k_{i,j+\frac{1}{2}}(u_{i,j+1}-u_{i,j}) - k_{i,j-\frac{1}{2}}(u_{i,j}-u_{i,j-1}) \right] = f_{i,j}$$
and get
$$a_P u_{i,j} - a_E u_{i+1,j} - a_W u_{i-1,j} - a_N u_{i,j+1} - a_S u_{i,j-1} = h^2 f_{i,j}$$
which
\begin{itemize}
  \item $a_E = k_{i+1/2, j}$
  \item $a_W = k_{i-1/2, j}$
  \item $a_N = k_{i, j+1/2}$
  \item $a_S = k_{i, j-1/2}$
  \item $a_P = a_E + a_W + a_N + a_S$
\end{itemize}

\subsection*{Code}
Shown in the attachment.

\clearpage

\subsection*{Final Result}

\textbf{Convergence speed comparison:}

\begin{figure}[h]
  \centering
  \includegraphics[width=.8\textwidth]{../plots/Convergence_Comparison.jpg}
  \caption{Convergence Comparison}
  \label{Convergence_Comparison}
\end{figure}

\begin{table}[htbp]
  \centering
  \label{tab:convergence_results}
  \begin{tabular}{lcc}
    \toprule
    \textbf{Method} & \textbf{Iterations} & \textbf{Time (s)} \\
    \midrule
    Line SOR ($\omega=1.75$) & 133 & 0.5103 \\
    Conjugate Gradient (CG) & 70 & 0.0025 \\
    Multigrid (V-Cycle) & \textbf{5} & 0.5535 \\
    \bottomrule
  \end{tabular}
  \caption{Comparison of numerical methods for solving the electrostatic field problem ($N=64$, relative tolerance $\varepsilon = 10^{-6}$).}
\end{table}

\begin{itemize}
  \item \textbf{Multigrid (MG)}: Exhibits the fastest convergence.
    For elliptic equations, MG typically demonstrates a grid-independent convergence rate (requiring $O(1)$ iterations).
    It generally achieves a relative error of $10^{-6}$ within 5 iterations.
  \item \textbf{Conjugate Gradient (CG):} Exhibits moderate convergence speed.
    For a grid with $N=64$, the condition number is $\kappa(A) \approx O(N^2)$,
    and the number of iterations for CG is roughly proportional to $N$ (specifically $\sqrt{\kappa}$).
    While significantly faster than SOR, it is slower than MG.
  \item \textbf{Line Gauss-Seidel SOR:} Exhibits the slowest convergence (despite utilizing an optimized relaxation factor $\omega$).
    As the iteration count increases, once high-frequency errors are eliminated, the removal of low-frequency errors becomes extremely slow.
\end{itemize}

\end{document}
